\documentclass[12pt]{article}

\usepackage{a4}
\usepackage{graphicx}
\usepackage{fullpage}
\usepackage{microtype}
\usepackage{pslatex}
% \setlength{\parskip}{0.5em}
% \setlength{\parindent}{4em}
% \hbadness=99999

\title{Tugas Agama Kristen Protestan \\ \textbf{"Keadilan menurut Alkitab"}}
\date{15 Januari 2020}
\author{Rivo Juicer Wowor - XII Multimedia}

\begin{document}
    \maketitle
    \section{Pilihan Ganda}
    \begin{enumerate}
        \item Semua binatang ada pasangannya. Tetapi, Adam tidak menemukan pasangan yang sepadan.
            Tindakan adil Allah dalam konteks ini adalah \dots
            \begin{enumerate}
                \item \textbf{memberikan pasangan yaitu Hawa}
                \item memberi kuasa atas semua ciptaan
                \item melindungi dan menyertai manusia
                \item menyediakan semua kebutuhan Adam
                \item memberkati hidup dengan berkelimpahan
            \end{enumerate}
        \item Adil adalah salah satu sifat Allah. Karena keadilan-Nya itu, tindakan Allah 
            terhadap orang berdosa adalah \dots
            \begin{enumerate}
                \item membenci
                \item membiarkan
                \item menghukum
                \item \textbf{menghakimi}
                \item mengacuhkan
            \end{enumerate}
        \item Allah tidak pandang bulu, seruan Yesus, \textit{"Eli, Eli, lamasabakhtani"}
            mengandung makna keadilan, yaitu \dots
            \begin{enumerate}
                \item bersama Allah pada saat disalibkan
                \item Allah tidak berkenan terhadap Yesus
                \item Allah tidak dapat berkompromi dengan dosa 
                \item dosa telah memisahkan manusia dari kasih Allah
                \item \textbf{Yesus tidak mendatangkan hukuman karena dosa}
            \end{enumerate}
        \item Keadilan Allah yang tidak pernah berubah dinyatakan 
            dengan kemurahan-Nya yaitu \dots
            \begin{enumerate}
                \item \textbf{pengampunan kepada orang yang bertobat}
                \item penghakiman pada saat kedatangan Yesus
                \item manusia akan terbagi ke surga dan neraka
                \item penghukuman bagi orang yang berdosa
                \item Tuhan Yesus bisa kembali ke surga
            \end{enumerate}
        \item Allah menghukum bangsa Israel karena penyembahan berhala yang mereka lakukan,
            sebab bangsa itu telah \dots
            \begin{enumerate}
                \item melakukan perubahan
                \item mengalami kemajuan
                \item meniru bangsa lain
                \item mengikuti tren
                \item \textbf{berubah setia}
            \end{enumerate}
        \item Keadilan tidak terlepas dari kesetiaan. Allah menghendaki
            agar umat-Nya hidup dengan cara \dots
            \begin{enumerate}
                \item menaati raja-rajanya
                \item memusuhi bangsa lainnya
                \item \textbf{melakukan Taurat-Nya}
                \item mengutamakan keluarga
                \item menyebar ke segala tempat
            \end{enumerate}
        \item Ketidaksetiaan manusia kepada Allah menggambarkan sifat manusia berdosa yaitu \dots
            \begin{enumerate}
                \item menyukai ketidakadilan
                \item cenderung rendah diri
                \item \textbf{sombong dan angkuh}
                \item mudah dipengaruhi
                \item menjauh dari Allah
            \end{enumerate}
        \item Adanya surga dan neraka mencerminkan bahwa kasih Allah kepada manusia adalah \dots
            \begin{enumerate}
                \item hanya untuk umat-Nya
                \item \textbf{kasih yang berkeadilan}
                \item kasih yang mengampuni
                \item membenci manusia berdosa
                \item sempurna dan tak terjangkau manusia
            \end{enumerate}
        \item Respons yang benar terhaadap keadilan Allah adalah \dots
            \begin{enumerate}
                \item dapat menjalankan hukum 
                \item memiliki nilai-nilai keadilan 
                \item bersedia menjalani hukuman 
                \item \textbf{ketaatan terhadap hukum Allah}
                \item mempelajari semua hukum Allah 
            \end{enumerate}
        \item Peristiwa Izebel yang merebut kebun anggur milik nabot tidak hanya
            melakukan kejahatan kepada rakyatnya, tetapi juga pelanggaran terhadap \dots
            \begin{enumerate}
                \item kewenangan Raja Ahab 
                \item hak rakyat akan hartanya 
                \item peraturan di antara bangsa
                \item \textbf{hukum Allah tentang warisan}
                \item asas hidup saling menghormati
            \end{enumerate}
        \item Ketidakadilan dalam keluarga juga terjadi pada saat \dots
            \begin{enumerate}
                \item kasih Ishak kepada Esau
                \item Kain menerima hukuman Allah
                \item \textbf{pembagian tanah menurut kaum}
                \item Esau menjual hak kesulungannya
                \item Sara melahirkan Ishak sesuai janji Allah 
            \end{enumerate}
        \item Daud mengajukan keadilan kepada rakyatnya ketika ia melakukan tindakan \dots
            \begin{enumerate}
                \item membawa Mefiboset ke istananya
                \item meratapi anaknya yang meninggal
                \item merebut Batsyeba menjadi permaisuri
                \item mengasihi Absalom yyang sangat tampan
                \item \textbf{menyediakan bahan bangunan Bait Allah}
            \end{enumerate}
        \item Remaja Kristen yang memiliki nilai-nilai keadilan dalam hidupnya pasti \dots
            \begin{enumerate}
                \item mengembalikan buku milik temannya
                \item mengganti kerugian karena kelalaiannya
                \item \textbf{memenuhi panggilan ke sekolah ketika bersalah}
                \item membayar tagihan yang menjadi tanggung jawabnya
                \item melaksanakan kewajiban agar hak orang lain terpenuhi
            \end{enumerate}
        \item Renny dituduh menyebarkan berita kebohongan meskipun ia tidak melakukannya.
            Tindakan Renny yang mencerminkan keadilan adalah \dots
            \begin{enumerate}
                \item \textbf{melapor kepada yang berwajib}
                \item membela diri di depan umum
                \item mengumpulkan semua bukti
                \item berdiam diri dan berdoa 
                \item menjauh dari pergaulan
            \end{enumerate}
        \item Ketika teman dekatmu berkhianat, yang harus kamu lakukan sebagai remaja Kristen
            adalah \dots
            \begin{enumerate}
                \item menjauh dari persahabatan
                \item mendiamkan perbuatannya
                \item \textbf{menegur dan menasihatinya}
                \item memutus komunikasi sementara
                \item melaporkan kepada orangtuanya
            \end{enumerate}
        \item Tuhan Yesus mengajarkan keadilan kepada murid-murid-Nya, yaitu \dots
            \begin{enumerate}
                \item \textbf{mengampuni orang lain tanpa batas setiap saat}
                \item memberitahu bahwa Petrus akan menyangkal
                \item memberikan apa yang menjadi hak seseorang
                \item mengasihi dengan cara menaati-Nya
                \item menanti datangnya Roh Kudus
            \end{enumerate}
        \item Allah menyediakan penghukuman tetapi juga pengampunan. Itu sebabnya Allah 
            telah menyediakan jalan pendamaian, yaitu \dots
            \begin{enumerate}
                \item \textbf{penebusan Kristus}
                \item hukuman sesuai Taurat
                \item hakim di antara orang Israel
                \item imam untuk upacara Yahudi
                \item korban untuk upacara pendamaian
            \end{enumerate}
        \item Bukti keadilan Allah kepada semua makhluk ciptaan-Nya adalah \dots
            \begin{enumerate}
                \item semua orang dapat membaca Alkitab
                \item memberi sakit kepada semua manusia
                \item memberi Hukum Sepuluh kepada semua orang
                \item \textbf{semua makhluk menikmati udara dan matahari}
                \item menetapkan hukuman yang sama di seluruh dunia
            \end{enumerate}
        \item Seorang ayah menasihati agar anaknya menaati firman Tuhan. Yang perlu
            dilakukan oleh ayah adalah \dots
            \begin{enumerate}
                \item membuat kesepakatan untuk menaati Allah
                \item mengawasi agar anaknaa menaati nasihat
                \item \textbf{meneladankan cara menaati firman Allah}
                \item menyepakati hukuman jika melanggar
                \item menjanjikan hadiath atas ketaatannya
            \end{enumerate}
        \item 
    \end{enumerate}
\end{document}