% Options for packages loaded elsewhere
\PassOptionsToPackage{unicode}{hyperref}
\PassOptionsToPackage{hyphens}{url}
%
\documentclass[
]{article}
\usepackage{lmodern}
\usepackage{amssymb,amsmath}
\usepackage{ifxetex,ifluatex}
\ifnum 0\ifxetex 1\fi\ifluatex 1\fi=0 % if pdftex
  \usepackage[T1]{fontenc}
  \usepackage[utf8]{inputenc}
  \usepackage{textcomp} % provide euro and other symbols
\else % if luatex or xetex
  \usepackage{unicode-math}
  \defaultfontfeatures{Scale=MatchLowercase}
  \defaultfontfeatures[\rmfamily]{Ligatures=TeX,Scale=1}
\fi
% Use upquote if available, for straight quotes in verbatim environments
\IfFileExists{upquote.sty}{\usepackage{upquote}}{}
\IfFileExists{microtype.sty}{% use microtype if available
  \usepackage[]{microtype}
  \UseMicrotypeSet[protrusion]{basicmath} % disable protrusion for tt fonts
}{}
\makeatletter
\@ifundefined{KOMAClassName}{% if non-KOMA class
  \IfFileExists{parskip.sty}{%
    \usepackage{parskip}
  }{% else
    \setlength{\parindent}{0pt}
    \setlength{\parskip}{6pt plus 2pt minus 1pt}}
}{% if KOMA class
  \KOMAoptions{parskip=half}}
\makeatother
\usepackage{xcolor}
\IfFileExists{xurl.sty}{\usepackage{xurl}}{} % add URL line breaks if available
\IfFileExists{bookmark.sty}{\usepackage{bookmark}}{\usepackage{hyperref}}
\hypersetup{
  hidelinks,
  pdfcreator={LaTeX via pandoc}}
\urlstyle{same} % disable monospaced font for URLs
\setlength{\emergencystretch}{3em} % prevent overfull lines
\providecommand{\tightlist}{%
  \setlength{\itemsep}{0pt}\setlength{\parskip}{0pt}}
\setcounter{secnumdepth}{-\maxdimen} % remove section numbering
\ifluatex
  \usepackage{selnolig}  % disable illegal ligatures
\fi

\author{}
\date{}

\begin{document}

Pada bulan Agustus lalu, PT Visi Citra Mitra Mulia dan PT Rajawali Citra
Televisi Indonesia menggugat UU Nomor 32 Tahun 2002 tentang Penyiaran ke
Mahkamah Konstitusi. Kedua perusahaan media ini mengajukan uji materi
soal UU Penyiaran dan menilai Pasal 1 angka 2 UU Penyiaran menyebabkan
perlakuan berbeda antara penyelenggara penyiaran konvensional yang
menggunakan frekuensi radio dengan penyelenggara penyiaran \emph{over
the top (OTT)} yang menggunakan internet seperti \emph{YouTube} dan
\emph{Netflix}. Jika gugatan ini dikabulkan, maka masyarakat baik
perorangan maupun badan usaha terancam tidak leluasa menggunakan media
sosial, seperti \emph{YouTube, Instagram, Facebook}, dan sebagainya
untuk melakukan siaran langsung.

Hal ini tentunya membuat banyak masyarakat yang emosi dan kecewa
terhadap kedua perusahaan ini. Karena banyak orang yang menjadikan
layanan ini sebagai sumber penghasilan bagi hidup mereka. Jika gugatan
ini disetujui, maka otomatis sumber penghasilan mereka akan dipotong
total serta menghambat pertumbuhan ekonomi kreatif dan digital di
Indonesia. Selain itu, hak kebebasan berbicara juga akan dipertaruhkan
disini. Karena layanan ini merupakan salah satu media masyarakat yang
digunakan untuk mengekspresikan pendapat dan opini mereka.

Gugatan ini juga dapat dibantahkan dengan pernyataan bahwa layanan
siaran langsung seperti \emph{YouTube} dan lain-lain tunduk pada UU
Telekomunikasi, bukan UU Penyiaran. Karena cara beroperasi layanan ini
jauh berbeda dengan layanan televisi konvensional, meskipun keduanya
sama-sama menggunakan audio dan visual sebagai medianya.

Sebaiknya, layanan siaran langsung \emph{OTT} ini dilindungi oleh sebuah
lembaga yang menaungi layanan media \emph{OTT} agar kegiatan siaran
langsung dapat diawasi penggunaannya. Selain itu, undang-undang baru
seharusnya dibuat untuk mengatur layanan siaran melalui internet. Agar
layanan-layanan ini menjadi legal untuk dipakai di Indonesia. Karena
jika gugatan seperti ini dikabulkan, maka akan sangat berdampak pada
masa depan ekonomi kreatif Indonesia. Kita tidak tahu gugatan seperti
apa lagi yang akan datang yang berkaitan dengan layanan Internet yang
ada di Indonesia.

\end{document}
