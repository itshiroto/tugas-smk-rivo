\documentclass[12pt,a4paper]{report}

\usepackage{a4}
\usepackage{fullpage}
\usepackage[bahasa]{babel}
\usepackage{titlesec}
\usepackage[hyphenbreaks]{breakurl}
\usepackage[hyphens]{url}
\usepackage{graphicx}

\hbadness=99999

\setlength{\parskip}{0.7em}
\setlength{\parindent}{0em}

\begin{document}
  \section*{Software Manipulasi Audio}\label{software-manipulasi-audio}

  \subsection*{Audacity}\label{audacity}

  \includegraphics[width=\textwidth]{images/audacity.png}
  Audacity adalah aplikasi manipulasi audio gratis dan \emph{open source} (satu-satunya di list
  ini) untuk sistem operasi Windows, Mac, dan Linux.

  Audacity memiliki beberapa fitur dasar untuk memanipulasi audio,
  seperti:

  \begin{itemize}
  \item
    Audio Recording
  \item
    Multitrack Recording and Editing
  \item
    Scrubbing
  \item
    MIDI Playback
  \item
    Penggunaan plugin VST (hanya terbatas untuk VST Effect, bukan MIDI)
  \item
    Analisa Spektrum Audio
  \item
    dsb
  \end{itemize}

  Audacity unggul di harga dan fleksibilitasnya. Sebagai satu-satunya
  aplikasi yang gratis di list ini serta UI yang mudah digunakan, Audacity
  banyak dipakai oleh banyak Audio Engineer pemula dan juga pembawa
  podcast di seluruh dunia. Hanya saja, kekurangan Audacity dapat dilihat
  dari kurangnya support untuk driver ASIO, Perekaman MIDI, dan juga
  tampilan UI-nya yang terbilang \emph{jadul} dan tidak profesional.

  \subsection*{Adobe Audition}\label{adobe-audition}

  \includegraphics[width=\textwidth]{images/audition.jpg} Adobe Audition merupakan salah
  satu dari keluarga aplikasi Adobe Creative Cloud yang dimiliki oleh
  Adobe. Audition dapat dijalankan baik di Windows maupun Mac.

  Beberapa fitur unggulan Adobe Audition antara lain,

  \begin{itemize}
  \item
    Analisis Frekuensi Audio secara langsung
  \item
    Dynamic Link
  \item
    Project Manager
  \item
    Mengedit audio dalam video
  \end{itemize}

  Adobe Audition banyak dipakai oleh audio engineer di dunia karena
  Audition memiliki integrasi yang sangat kuat dengan aplikasi Adobe
  lainnya, seperti Premiere Pro, After Effects, dll. Sayangnya, Adobe
  Audition memiliki masalah yang kurang lebih sama seperti software
  lainnya, yaitu tidak stabil (sering crash dan hang). Selain itu,
  Audition juga dipatok dengan harga yang cukup mahal dan menggunakan
  pembayaran dengan cara berlangganan, sehingga tidak semua orang dapat
  membelinya.

  \subsection*{Ableton Live}\label{ableton-live}

  \includegraphics[width=\textwidth]{images/ableton.jpg}
  Ableton Live adalah aplikasi Digital Audio Workstation yang dapat dijalankan di Windows dan Mac.
  Aplikasi DAW seperti ini biasanya ditargetkan untuk kalangan produser
  musik untuk membuat musik, tapi banyak juga yang memakai DAW sebagai
  aplikasi recording dan manipulasi audio biasa.

  Fitur unggulan Ableton Live antara lain,

  \begin{itemize}
  \item
    UI yang sederhana
  \item
    Kemampuan untuk menggunakan Loop dan Sample
  \item
    Banyaknya \emph{built-in effects} yang disediakan
  \end{itemize}

  Ableton Live lebih ditargetkan kepada musisi dan produser hip-hop dan
  electronic karena kemampuan looping dan samplingnya yang lebih kuat
  dibanding aplikasi lainnya. Hanya saja, harga dari Ableton Live
  tergolong sangat mahal meskipun \emph{one-time purchase}. Selain itu,
  meskipun Ableton memiliki UI yang sederhana, tapi tampilannya dapat
  mengintimidasi pengguna baru dan orang awam.

  \subsection*{Logic Pro}\label{logic-pro}

  \includegraphics[width=\textwidth]{images/logic.jpg}
  Logic Pro adalah aplikasi DAW dari Apple yang hanya bisa digunakan di Mac saja. Aplikasi ini merupakan
  salah satu standar industri (selain Pro Tools) yang banyak dipakai di
  studio terkenal di seluruh dunia.

  Beberapa fitur unggulan Logic Pro antara lain,

  \begin{itemize}
  \item
    Performa yang lebih baik
  \item
    Tampilan UI yang menyerupai peralatan
    Analog tapi tetap mempertahankan modernitas
  \item
    Memiliki fitur audio recording lebih lengkap ketimbang yang lainnya
  \end{itemize}

  Karena dibuat dan didesain oleh Apple, maka tentu performa yang
  dihasilkan dari aplikasi ini tentu sangat responsif ketika digunakan.
  Tapi \emph{trade-off}nya adalah Logic hanya bisa digunakan pada sistem
  Mac saja. Logic juga memiliki kemampuan analog/audio recording yang
  lebih baik ketimbang kompetitor lainnya tapi hal itu juga ditukar dengan
  kemampuan recording MIDI dsb yang kurang.

  \subsection*{FL Studio}\label{fl-studio}

  \includegraphics[width=\textwidth]{images/flstudio.jpg}
  FL Studio merupakan aplikasi DAW yang dikembangkan oleh Image-line dan dapat digunakan di Windows dan
  Mac. Aplikasi ini terkenal di kalangan produser Indonesia karena
  banyaknya tutorial di Youtube dan juga tampilan yang mudah digunakan.

  Fitur unggulan dari FL Studio antara lain,

  \begin{itemize}
  \item
    UI yang sederhana dan modern
  \item
    Tersedia banyak tutorial dan pelatihan di internet
  \item
    Sequencer yang intuitif
  \end{itemize}

  FL Studio memiliki UI yang sangat mudah digunakan sehingga tidak heran
  banyak orang di Indonesia yang menggunakan aplikasi ini. Selain itu,
  \emph{built-in effects} yang cukup banyak serta mudah digunakan
  merupakan salah satu poin pendukung aplikasi ini. Tapi FL Studio
  memiliki workflow yang menurut saya sedikit \emph{clunky} dan juga
  kemampuan mixing yang menurut saya kurang fleksibel. Tapi mungkin banyak
  orang yang sudah terbiasa dan juga lebih tertarik terhadap workflow dan
  UI aplikasi ini. Selain itu, FL Studio juga memberikan \emph{Lifetime
  Free Upgrade} tidak seperti aplikasi kompetitor lainnya (kecuali
  Audacity dan Audition)

  \subsection*{Reaper}\label{reaper}

  \includegraphics[width=\textwidth]{images/reaper.jpg} Reaper sangat kuat dan kaya fitur
  juga relatif lebih terjangkau. Sebagai permulaan, Reaper hadir dengan
  dukungan untuk beberapa trek, dan memiliki dukungan~multichannel~yang
  luar biasa dengan 64 saluran di setiap trek.

  Reaper juga memiliki kemampuan untuk merekam~audio~langsung
  ke~mono,~stereo~atau bahkan file~audio~multikanal, bersama dengan
  kemampuan untuk merekam ke beberapa~disk~pada saat yang sama untuk
  redundansi data. Hanya saja,~user interface~dari Reaper tidak sebagus
  Adobe Audition dan Reaper agak sulit dipakai oleh pemula.

  \subsection*{Presonus Studio One}\label{presonus-studio-one}

  \includegraphics[width=\textwidth]{images/studioone.png} Presonus Studio One 4 adalah DAW
  serbaguna yang hadir dengan banyak fitur keren dan berguna. Ada dukungan
  untuk beberapa trek, dan dengan fitur Chord Track Studio One, sehingga
  bisa dengan mudah membuat prototipe cepat dari suatu lagu dan
  mendapatkan gagasan tentang seperti apa lagu itu terdengar.

  \emph{Chord Track} membawa fitur seperti modulasi kunci, substitusi akor
  dan lainnya untuk membuat protoyping lebih mudah. Studio One dapat
  secara otomatis mengidentifikasi chord dari track~audiomu. Namun untuk
  pemula software ini kurang cocok karena \emph{learning curve} yang cukup
  tinggi.

  \subsection*{Hindenburg Pro}\label{hindenburg-pro}

  \includegraphics[width=\textwidth]{images/hindenburg.jpg} Hindenburg Pro juga merupakan
  software editing audio yang ditargetkan kepada podcaster dan jurnalis.
  Software ini lintas platform dan bekerja dengan Windows dan macOS.
  Hindenburg Pro juga dapat mengimpor file audio 24-bit dan bahkan bekerja
  dalam sesi 24-bit. Selain itu, DAW membawa sejumlah besar efek termasuk
  kompresor, EQ, loudness meters, dan dukungan untuk plugin vST sehingga
  dapat memperluas pilihan efek yang diinginkan user.

  Hindenburg memiliki fitur-fitur andalan seperti EQ otomatis, volume
  normalization, de-bleed, dll yang tidak dimiliki oleh aplikasi
  kompetitor lainnya. Alur kerja Hindenburg ramping dan intuitif. Dapat
  mengimpor audio dari proyek lain, seperti audio dari perekam suara, atau
  dapat merekam langsung ke dalam software ini sendiri.
  \end{document}