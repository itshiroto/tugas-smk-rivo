\documentclass[12pt]{article}

\usepackage{a4}
\usepackage{fullpage}
\setlength{\parskip}{1em}
\setlength{\parindent}{4em}

\title{\textbf{Tahap Pascaproduksi}}
\author{Rivo Juicer Wowor\\ XII Multimedia, SMKN 1 Berau}
\date{}

\begin{document}
    \maketitle
    \pagenumbering{arabic}
    
    \section{Definisi Tahap Pascaproduksi}
    Tahap Pascaproduksi merupakan tahap terakhir dalam produksi video dan tahap sebelum penyajian video. Tahap ini memerlukan perangkat komputer dan perangkat lunak \textit{video editing} yang memadai. Banyak aplikasi yang dapat digunakan untuk mengedit video. Salah satu contoh yang paling banyak dipakai ialah \textsc{Adobe Creative Cloud.} Adobe Creative Cloud atau yang biasa disebut dengan Adobe CC merupakan keluarga dari aplikasi keluaran Adobe. Dalam Adobe CC sendiri, berisi aplikasi seperti \textsc{Adobe Photoshop, Adobe Illustrator, Adobe Premiere Pro, Adobe After Effects,} dan lainnya.

    \section{Tahap-tahap dalam Pascaproduksi}
    Adapun tahap pascaproduksi terbagi menjadi empat bagian, yaitu sebagai berikut:
    
    \subsection{\textit{Capturing/Importing}}
    Tahap ini merupakan proses memindahkan hasil rekaman dari kamera ke perangkat editing. Capturing dilakukan bila hasil rekaman tidak berupa file video, tetapi berbentuk \textit{reel}. Sedangkan importing dilakukan bila hasil rekaman berupa file video.

    \subsection{\textit{Offline Editing}}
    Tahap \textit{offline editing} merupakan tahap di mana pencatat script/scriptwriter bekerja saat produksi. Scriptwriter juga dapat mencatat kembali semua hasil shooting dan adegan serta menuliskan adegan tersebut ke dalam \textit{time code} yang ada di dalam aplikasi. Jadi dari catatan tersebut, dibuatlah hasil edit secara kasar yang disebut dengan \textit{offline editing}.
    
    \subsection{\textit{Online Editing}}
    Setelah tahap pertama dilakukan dan dirasa kurang maksimal atau sesuai. Maka dapat dilakukan \textit{\textbf{Online Editing}}. Berdasarkan script editing yang sudah dibuat pada tahapan \textit{offline editing}, editor selanjutnya dapat melakukan pengeditan yang lebih rapi lagi. Editor dapat melihat adegan per adegan dan shoot per shoot untuk menyatukan cerita agar berkesinambungan serta memperindah hasil pengeditan video/\textit{post-production}.

    \subsection{Pemotongan}
    Tahap paling penting dimana kita memotong video mentah dan mengambil hasil potongan yang dirasa lebih baik.

    \subsection{Pengaturan Transisi}
    Transisi merupakan bentuk perpindahan antar potongan video/clip untuk menjaga kelanjutan sebuah video, membentuk suasana, dan lainnya. Jenis-jenis transisi adalah:
    \begin{enumerate}
        \item Cut/Cut to
        \item Dissolve
        \item Wipe
        \item Fade
        \item Pemanduan Suara
    \end{enumerate}

    \subsection{\textit{Mixing}}
    Proses \textit{mixing} merupakan proses menggabungkan atau menyinkronisasikan antara video dan audio. Dalam tahap ini, proses mixing mengutamakan sound design dari video yang dikerjakan. Sound design yang dimaksud disini adalah pemrosesan audio dalam video seperti penggabungan audio, penambahan SFX dan BGM, serta lainnya. Hal ini dilakukan untuk membangun suasana dalam video.

    \subsection{Rendering}
    Rendering merupakan proses terakhir dalam penyatuan hasil editing menjadi satu kesatuan video yang utuh.

    \subsection{\textit{Preview}}
    Tahap terakhir yang dilakukan adalah tahap \textit{Preview}. Tahap ini merupakan \textit{screening} akhir dalam melihat video yang sudah selesai diedit. Setelah semua pihak setuju dengan hasil editnya, maka dilakukan proses mastering. Proses mastering ini merupakan tahap terakhir setelah preview. Dimana tahap ini merupakan proses gabungan semua elemen video untuk dijadikan sebuah file. File tersebut dapat berupa file mentah maupun video yang sudah siap seperti kepingan DVD master.


    \subsection{Distribusi dan Pengiklanan Film}
    Proses distribusi video merupakan proses penayangan film yang telah dibuat. Caranya bisa dilakukan melalui cara online seperti meng-\textit{upload}nya ke Youtube, atau cara offline seperti premiering pada salah satu bioskop. Lalu ada proses pengiklanan film, dimana proses ini bertujuan untuk mempromosikan karya film yang telah dibuat. Sama seperti distribusi, proses ini bisa dilakukan secara Online seperti mempromosikannya melalui sosial media, atau offline seperti mengadakan suatu kegiatan promosi seperti konferensi pers yang berisi tentang perilisan film.


\end{document}