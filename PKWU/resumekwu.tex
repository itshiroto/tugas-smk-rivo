\hypertarget{pengertian-wirausaha}{%
\subsection{Pengertian Wirausaha}\label{pengertian-wirausaha}}

Kewirausahaan merupakan semangat, sikap, perilaku, dan kemampuan
seseorang untuk menangani usaha atau kegiatan yang mengarah pada upaya
mencari, menciptakan serta menerapkan cara kerja, teknologi, dan produk
baru dengan meningkatkan efisiensi dalam rangka memberikan pelayanan
yang lebih baik atau memperoleh keuntungan yang lebih besar. Dapat
disimpulkan bahwa Wirausaha adalah kemampuan dan tindakan untuk
menciptakan sesuatu yang dapat memberikan peluang dan nilai tambah.

Orang yang menjalankan atau melaksanakan tindakan kewirausahaan disebut
sebagai Wirausahawan. Seorang wirausahawan juga berani mengambil risiko
dengan membuat pembaruan atau inovasi, serta belajar akan kegagalan.

\hypertarget{tujuan-wirausaha}{%
\subsection{Tujuan Wirausaha}\label{tujuan-wirausaha}}

\begin{itemize}
\tightlist
\item
  Menciptakan peluang kerja, mempermudah hidup orang, dsb.
\item
  Menciptakan nilai tambah dan inovasi baru.
\item
  Memadukan sumber daya dan merealisasikannya
\end{itemize}

\hypertarget{faktor-yang-mendorong-wirausahawan}{%
\subsection{Faktor yang mendorong
wirausahawan}\label{faktor-yang-mendorong-wirausahawan}}

\begin{enumerate}
\def\labelenumi{\arabic{enumi}.}
\tightlist
\item
  Untuk mencapai kebutuhan yang lebih baik/prestasi.
\item
  Kebutuhan untuk tidak bergantung pada orang lain.
\item
  Kebutuhan akan pembaharuan/inovasi.
\item
  Keinginan untuk mendapatkan pendapatan yang memadai.
\item
  Keinginan untuk hidup sejahtera
\item
  Keinginan untuk mengerjakan sesuatu berdasarkan passion.
\end{enumerate}

\hypertarget{karakteristik-kewirausahaan}{%
\subsection{Karakteristik
kewirausahaan:}\label{karakteristik-kewirausahaan}}

\begin{enumerate}
\def\labelenumi{\arabic{enumi}.}
\tightlist
\item
  Motivasi yang tinggi untuk memenuhi kebutuhan hidup
\item
  Berorientasi ke masa depan
\item
  Jiwa kepemimpinan yang unggul
\item
  Jaringan usaha yang luas
\item
  Tanggap dan kreatif menghadapi perubahan
\end{enumerate}

\hypertarget{inspirasi-wirausahawan-indonesia-dalam-teknologi-digital}{%
\subsection{Inspirasi wirausahawan Indonesia dalam teknologi
digital}\label{inspirasi-wirausahawan-indonesia-dalam-teknologi-digital}}

\begin{enumerate}
\def\labelenumi{\arabic{enumi}.}
\item
  William Tanuwijaya

  Pendiri dari Tokopedia serta salah satu pelopor startup di Indonesia.
  Awalnya ia susah untuk mendapatkan investor karna banyak dari mereka
  yang tidak tertarik pada bisnis marketplace. Tapi pada akhirnya
  Tokopedia menjadi perusahaan yang paling banyak digunakan pada
  bidangnya di Indonesia.
\item
  Ferry Unardi

  Pendiri dari Traveloka pada tahun 2012 yang kemudian menjadi
  perusahaan startup ketiga di Indonesia yang bergelar unicorn. Perlu
  waktu 5 tahun bagi Traveloka untuk mendapatkan status unicorn setelah
  berhasil mendapat investasi senilai USD500 juta dari banyak perusahaan
  perjalanan online terkemuka dunia.
\item
  Ahmad Zaky

  Pendiri dari Bukalapak. Dibangun pada tahun 2010, Bukalapak bertujuan
  untuk mengembangkan UMKM di Indonesia dan akhirnya menjadi startup
  keempat yang menyandang gelar Unicorn.
\item
  Nadiem Makarim

  Pendiri dari Gojek, dan merupakan startup pertama yang menyandang
  gelar unicorn pada tahun 2016. Perlu waktu 6 tahun bagi Nadiem dan
  Gojek untuk dapat menjadi unicorn setelah lahir di tahun 2010. Status
  unicorn itu disematkan setelah Gojek menerima suntukan dana senilai
  USD550 juta dari berbagai investor.
\end{enumerate}
