\documentclass[12pt,a4paper]{report}

\usepackage{a4}
\usepackage{fullpage}
\usepackage[bahasa]{babel}
\usepackage{titlesec}
\usepackage[hyphenbreaks]{breakurl}
\usepackage[hyphens]{url}
\usepackage[superscript,biblabel]{cite}

\titleformat{\chapter}[display] {\normalfont\Huge\bfseries}{\chaptertitlename\ \thechapter}{0pt}{\Huge}
\titlespacing{\chapter}{0cm}{1em}{1em}

\hbadness=99999

\setlength{\parskip}{1em}
\setlength{\parindent}{0em}

\addto{\captionsbahasa}{%
    \renewcommand{\bibname}{Daftar Pustaka}}

\title{Makalah Industri Kreatif PKWU}
\author{Rivo Juicer Wowor\\ XII Multimedia, SMKN 1 Berau}
\date{}

\begin{document}
    \maketitle
    \pagenumbering{arabic}

    \chapter{Pendahuluan}
    \section{Latar Belakang}
    \par Ekonomi Kreatif adalah sebuah konsep yang menempatkan kreativitas dan pengetahuan sebagai aset utama dalam menggerakkan ekonomi. Ekonomi Kreatif memiliki kaitan yang erat dengan industri kreatif. Ekonomi kreatif membutuhkan keberadaan industri kreatif dengan kreativitas sumber daya manusia sebagai aset utama untuk menciptakan nilai tambah ekonomi.
    \section{Rumusan Masalah}
    \begin{enumerate}
        \item Subsektor manakah yang menarik minat penulis?
        \item Industri Kreatif apa yang akan dibuat dari subsektor tersebut?
        \item Bagaimanakah perencanaannya?
    \end{enumerate}
    \section{Tujuan}
    \subsection{Tujuan Penelitian}
    \begin{enumerate}
        \item Untuk memahami beberapa subsektor industri kreatif yang\\kompeten di era sekarang.
        \item Memberikan konsep usaha yang bisa digunakan ketika berwirausaha nanti.
    \end{enumerate}
    \subsection{Kegunaan Penelitian}
    Hasil penelitian ini diharapkan bisa menambah wawasan pembaca dalam industri kreatif yang berkembang di Indonesia sekarang serta memberikan beberapa ide dari konsep yang akan ditulis oleh penulis.

    \chapter{Isi}
    \section{Subsektor Industri Kreatif}
    Di dalam industri kreatif, terdapat banyak sekali subsektor yang berbeda. Pada buku \emph{Produk Kreatif dan Kewirausahaan C3} sendiri menuliskan 16 subsektor yang ada dalam industri kreatif\cite{kwu}. Beberapa di antaranya yaitu \emph{Arsitektur, Fashion, Desain Interior, Desain Komunikasi Visual, Kuliner,} dan lainnya.

    Di antara 16 subsektor tersebut, 3 sektor yang menarik perhatian saya. Yaitu:
    
    \begin{enumerate}
        \item
        \textbf{Aplikasi dan Pengembangan Permainan}
        
        Sektor Aplikasi dan Pengembangan Permainan merupakan salah satu sektor yang sedang naik daun akhir-akhir ini. Angka penjualan konsol serta PC dan juga besarnya komunitas game yang ada di dunia membuat saya tertarik dengan sektor ini.

        \item
        \textbf{Film, Animasi, dan Video}

        Sektor Film, Animasi, dan Video merupakan salah satu sektor yang sudah lumayan lama berkembang. Dari era 1920 hingga sekarang, sektor ini selalu sukses setiap dekadenya. Dan kebetulan, Jurusan yang saya ikuti di SMK juga mempelajari perfilman sehingga membuat saya tertarik pada sektor ini.

        \item
        \textbf{Musik}
        
        Sama seperti sektor Film, sektor Musik juga sudah lama berkembang. Tiap era memiliki ciri khasnya masing-masing seperti \emph{Roaring 20's Jazz} pada tahun 1920 serta \emph{Disco Music Revolution} pada tahun 1970-1980 membuat sektor ini salah satu sektor yang sukses di tiap eranya. Dikarenakan kesuksesan sektor ini serta latar belakang saya yang juga merupakan seorang musisi membuat saya tertarik pada sektor ini.
    \end{enumerate}

    \newpage
    \section{Ide Industri Kreatif}
    Dari ketiga subsektor tersebut, saya memiliki ide untuk membuat suatu \textit{circle}/studio indie yang membuat sebuah game. Di dalam game ini, kita bisa menaruh elemen-elemen cinematic yang biasa ada di sektor perfilman disertai dengan musik yang megah yang biasa ada di sektor musik.

    Mengapa pengembangan game? Seperti yang saya jelaskan tadi, industri game sekarang mengalami lonjakan yang sangat drastis dari pandemi ini. Dan juga, rilisnya konsol \textit{next-gen} seperti \emph{PlayStation 5} dan \emph{Xbox Series X} membantu lonjakan tersebut. Banyak juga studio indie yang sukses dari industri ini, contohnya seperti \textit{Toby Fox\cite{tobyf}, Mojang,} dan juga \textit{Studio MDHR}\cite{mdhr}. Berdasarkan data yang ditulis oleh Reuters, industri game sendiri memiliki keuntungan sebesar \emph{\$159,3 milyar} pada tahun 2020 dan diprediksi akan melewati titik \emph{\$200 milyar} pada tahun 2023\cite{reuters1}.

    Untuk gamenya sendiri, saya memiliki ide untuk membuat game eksplorasi yang mengedepankan cerita serta musiknya. Sumber inspirasi saya dari proyek ini ialah game \emph{God of War, Ni no Kuni, The Elder Scrolls: Skyrim} dan \emph{Tales of Series}.
    
    \begin{thebibliography}{9}
        
        \bibitem{kwu}
            Ayodya, Widya. \textit{Produk Kreatif dan Kewirausahaan C3}. Jakarta: Penerbit Erlangga, 2019.

        \bibitem{tobyf}
            \textit{"'Undertale' Creator Toby Fox Opens Up About The Game's Massive Popularity"}, Cameron Koch.
            \url{https://venturebeat.com/2016/09/14/undertales-pc-indie-success-brought-its-creator-fear-guilt-and-gratitude/}

        \bibitem{mdhr}
            \textit{"Here's how the family behind the indie video game 'Cuphead' sold 3 million copies without any formal training in game design"}, Kevin Webb.
            \url{https://www.businessinsider.com/cuphead-origin-story-studio-mdhr-red-bull-gaming-2019-1}

        \bibitem{reuters1}
            \textit{"Report: Gaming revenue to top \$159B in 2020"}, Reuters. \url{https://www.reuters.com/article/esports-business-gaming-revenues/report-gaming-revenue-to-top-159b-in-2020-idUSFLM8jkJMl}       

    \end{thebibliography}
\end{document}

